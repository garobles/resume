%%%%%%%%%%%%%%%%%%%%%%%%%%%%%%%%%%%%%%%
% Deedy - One Page Two Column Resume
% LaTeX Template
% Version 1.2 (16/9/2014)
%
% Original author:
% Debarghya Das (http://debarghyadas.com)
%
% Original repository:
% https://github.com/deedydas/Deedy-Resume
%
% IMPORTANT: THIS TEMPLATE NEEDS TO BE COMPILED WITH XeLaTeX
%
% This template uses several fonts not included with Windows/Linux by
% default. If you get compilation errors saying a font is missing, find the line
% on which the font is used and either change it to a font included with your
% operating system or comment the line out to use the default font.
% 
%%%%%%%%%%%%%%%%%%%%%%%%%%%%%%%%%%%%%%
% 
% TODO:
% 1. Integrate biber/bibtex for article citation under publications.
% 2. Figure out a smoother way for the document to flow onto the next page.
% 3. Add styling information for a "Projects/Hacks" section.
% 4. Add location/address information
% 5. Merge OpenFont and MacFonts as a single sty with options.
% 
%%%%%%%%%%%%%%%%%%%%%%%%%%%%%%%%%%%%%%
%
% CHANGELOG:
% v1.1:
% 1. Fixed several compilation bugs with \renewcommand
% 2. Got Open-source fonts (Windows/Linux support)
% 3. Added Last Updated
% 4. Move Title styling into .sty
% 5. Commented .sty file.
%
%%%%%%%%%%%%%%%%%%%%%%%%%%%%%%%%%%%%%%%
%
% Known Issues:
% 1. Overflows onto second page if any column's contents are more than the
% vertical limit
% 2. Hacky space on the first bullet point on the second column.
%
%%%%%%%%%%%%%%%%%%%%%%%%%%%%%%%%%%%%%%


\documentclass[]{deedy-resume-openfont}
\usepackage{fancyhdr}
 
\pagestyle{fancy}
\fancyhf{}
 
\begin{document}

%%%%%%%%%%%%%%%%%%%%%%%%%%%%%%%%%%%%%%
%
%     TITLE NAME
%
%%%%%%%%%%%%%%%%%%%%%%%%%%%%%%%%%%%%%%
\namesection{Gabriel}{Robles}{
    \urlstyle{same}
    \href{https://www.linkedin.com/in/garobles}{linkedin.com/in/garobles} $\bullet$
    \href{https://www.github.com/garobles}{github.com/garobles} $\bullet$
    \href{mailto:garobles@uwaterloo.ca}{garobles@uwaterloo.ca} 
}

%%%%%%%%%%%%%%%%%%%%%%%%%%%%%%%%%%%%%%
%
%     COLUMN ONE
%
%%%%%%%%%%%%%%%%%%%%%%%%%%%%%%%%%%%%%%

\begin{minipage}[t]{0.33\textwidth} 

%%%%%%%%%%%%%%%%%%%%%%%%%%%%%%%%%%%%%%
%     EDUCATION
%%%%%%%%%%%%%%%%%%%%%%%%%%%%%%%%%%%%%%

\section{Education} 

\subsection{University of Waterloo}
\descript{4A - Software Engineering}
\location{Expected Graduation April 2022}
\sectionsep

%%%%%%%%%%%%%%%%%%%%%%%%%%%%%%%%%%%%%%
%     SKILLS
%%%%%%%%%%%%%%%%%%%%%%%%%%%%%%%%%%%%%%

\section{Programming}
\descript{Proficient}
C \textbullet{} C++ \textbullet{} Python \\
\descript{Experienced}
Java \textbullet{} Scala \textbullet{} Javascript \textbullet{} SQL \\
Shell \textbullet{} Assembly \textbullet{} OpenGL \textbullet{} OpenCV \\
Android \textbullet{} Matlab\\
\descript{Familiar}
HTML \textbullet{} CSS \textbullet{} JQuery \textbullet{} React \\
Go \textbullet{} Rails
\sectionsep

%%%%%%%%%%%%%%%%%%%%%%%%%%%%%%%%%%%%%%
%     Projects
%%%%%%%%%%%%%%%%%%%%%%%%%%%%%%%%%%%%%%

\section{Projects}
\descript{Puppet Pose}
A graphical application written in \textbf{C++} that allows to manipulate a 3D puppet. It uses
\textbf{OpenGL} to render the scene.
\sectionsep

\descript{Ray Tracer}
    An implementation of a Ray Tracer written in \textbf{C++} to render a scene. Implemented a second version in \textbf{CUDA} for parallelized execution.
\sectionsep

\descript{Lacs Mini Compiler}
    Compiler for a programming language subset of Scala (Lacs) written in \textbf{Scala}. It implements math operations, function calls, nested procedures, closures and a garbage collector.
\sectionsep

\descript{Tetris}
The classic tile-matching puzzle game written in \textbf{C++}. The game supports a \textbf{graphical} interface for X11; several levels of difficulty; and a customizable input system.
\sectionsep

\descript{PDF Reader}
An \textbf{Android} application written in \textbf{Java} to read, draw and highlight PDFs.
\sectionsep

\descript{Paint on Arduino}
A sketching program written in \textbf{C} for the \textbf{Arduino} platform. A touch LCD screen was used for I/O, and it allows to save the sketches as SVG files.
\sectionsep

%%%%%%%%%%%%%%%%%%%%%%%%%%%%%%%%%%%%%%
%
%     COLUMN TWO
%
%%%%%%%%%%%%%%%%%%%%%%%%%%%%%%%%%%%%%%

\end{minipage} 
\hfill
\begin{minipage}[t]{0.66\textwidth} 

%%%%%%%%%%%%%%%%%%%%%%%%%%%%%%%%%%%%%%
%     EXPERIENCE
%%%%%%%%%%%%%%%%%%%%%%%%%%%%%%%%%%%%%%

\section{Experience}
\runsubsection{Side Effects Software Inc} 
\descript{3D Software Developer}
\location{January 2021 - April 2021 | Toronto , ON}
\vspace{\topsep} % Hacky fix for awkward extra vertical space
\begin{tightemize}
    \item Researched and prototyped a novel method to deform volumetric data, iterating over different algorithms to obtain the most performant and precise one
    \item Discussed the features with the technical director and implemented the chosen algorithm in \textbf{C++}
    \item Used  the \textbf{tbb} library to parallelize the execution of the algorithm, resulting in performance improvement of 600\%.
\end{tightemize}
\location{September 2019 – December 2019 | Toronto, ON}
\begin{tightemize}
    \item Implemented a physically based sculpting tool using \textbf{Python} and Houdini's procedural nodes. This tool allows to modify meshes by applying grab, twist, scale or pinch brushes
    \item Implemented a user interface for the sculpting tool using \textbf{Python} to allow users to interactively modify polygon meshes
    \item Improved performance by \textbf{profiling} the tool reducing calculation time by 70\%
    %\item Documented the sculpting tool to ease its introduction to CG artists
\end{tightemize}
\sectionsep

\runsubsection{Derivative}
\descript{Graphics Engine Developer}
\location{May 2020 – August 2020 | Toronto, ON}
\begin{tightemize}
    \item Implemented several texture operators using \textbf{C++} and \textbf{CUDA} which expose \textbf{OpenCV} functionality to digital artists
    \item Implement a surface operator in \textbf{C++} which generates random points inside a mesh or on its surface. Apply several optimizations so the tool runs in realtime
    \item Implemented a code generation script in \textbf{Python} to generate boilerplate C++ code used to setup UI parameters
\end{tightemize}
\sectionsep

\runsubsection{University of Hawaii}
\descript{Research Assistant}
\location{January 2019 - April 2019 | Honolulu, HI}
\begin{tightemize}
    \item Created a GUI application, using \textbf{Visual C++} and the \textbf{MFC} library, to produce a variety of droplets shapes used for biosurface research
    \item Improved the ADSA \textbf{algorithm}, which calculates the surface tension of a liquid by analysing an image of a drop, such that it can work for incomplete droplets
    \item Replaced optimization library from minpack to dlib \textbf{doubling the speed} of the ADSA algorithm
    %\item Refactored and debugged legacy code, improving \textbf{readability} and \textbf{correctness}
\end{tightemize}
\sectionsep

\runsubsection{Behaviour Interactive}
\descript{Game Programmer for Deathgarden}
\location{May 2018 - August 2018 | Montreal, QC}
\begin{tightemize}
    \item Implemented game analytics using \textbf{Unreal Engine 4, C++, Docker, and Node.js} which helped game designers to balance the game
    %\item Designed, discussed, and implemented strategies to reuse code, in order to improve \textbf{scalability and maintainability}
    \item Programmed game components using \textbf{UE4} subsystems and its \textbf{networking} features, improving in-game communication
\end{tightemize}
\sectionsep

%%%%%%%%%%%%%%%%%%%%%%%%%%%%%%%%%%%%%%
%     AWARDS
%%%%%%%%%%%%%%%%%%%%%%%%%%%%%%%%%%%%%%

\section{Awards} 
\vspace{\topsep}
\begin{tightemize}
    \item “Becas para los Grupos De Alto Rendimiento”, Full Scholarship awarded to the top 0.1\% on the Ecuadorian standardized test.
    %\item \textbf{“Becas para los Grupos De Alto Rendimiento de Carreras Técnicas y Tercer Nivel”}, Full Scholarship awarded to the top 0.1\% on the Ecuadorian standardized test “ENES”
%    \item University of Waterloo President's Scholarship of Distinction, Awarded to students admitted with an admission average of 95\% or higher
    \item Term Dean's Honours List for four out of five academic terms.
\end{tightemize}
\sectionsep

%%%%%%%%%%%%%%%%%%%%%%%%%%%%%%%%%%%%%%
%     PUBLICATIONS
%%%%%%%%%%%%%%%%%%%%%%%%%%%%%%%%%%%%%%

\section{Publications} 
\renewcommand\refname{\vskip -1.5em} % Couldn't get this working from the .cls file
\bibliographystyle{abbrv}
\bibliography{./publications}
\nocite{*}

\end{minipage} 
\end{document}  \documentclass[]{article}
